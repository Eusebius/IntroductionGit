\documentclass[a4paper,twoside,french]{book}
\usepackage[T1]{fontenc}
\usepackage{graphicx}
\usepackage[french]{babel}
\usepackage{times}
\usepackage{url}
\usepackage[utf8]{inputenc}
\usepackage{amsmath}
\usepackage{latexsym}
\usepackage{xspace}
\usepackage[missing=Informations\ de\ version\ indisponibles]{gitinfo}
\sloppy

\addtolength{\topmargin}{-2cm}
\addtolength{\textheight}{+3cm}
\addtolength{\oddsidemargin}{-0,5cm}
\setlength{\evensidemargin}{\oddsidemargin}
\addtolength{\textwidth}{+3cm}

%\renewcommand\section{\@startsection {section}{1}{\z@}%
%	{-3.5ex \@plus -1ex \@minus -.2ex}%
%	{2.3ex \ at plus.2ex}%
%	{\reset@font\Large\bfseries}}

\makeatletter
\renewcommand\section{\@startsection
  {section}{1}{1cm}%name, level, indent
  {1.5\baselineskip}%             beforeskip
  {1\baselineskip}%            afterskip
  {\normalfont\Large\bfseries}}% style
  
\renewcommand\subsection{\@startsection
  {subsection}{2}{2cm}%name, level, indent
  {-\baselineskip}%             beforeskip
  {0.3\baselineskip}%            afterskip
  {\normalfont\normalsize\itshape\bfseries}}% style
  
\renewcommand\subsubsection{\@startsection
  {subsubsection}{3}{2.5cm}%name, level, indent
  {-\baselineskip}%             beforeskip
  {0.3\baselineskip}%            afterskip
  {\normalfont\normalsize\itshape\bfseries}}% style
\makeatother

%Tableau de versionning
\newenvironment{versions}%
    {\vspace{0cm}\begin{table}[b]\begin{flushright}\begin{tabular}{l@{\hspace{0.5cm}}c@{\hspace{0.5cm}}r}\hline}%
    {\end{tabular}\end{flushright}\end{table}}

\begin{document}
\pagestyle{empty}
\bibliographystyle{alpha}

\newcommand{\maintainer}{Guillaume Piolle}
\newcommand{\maintainerurl}{\url{http://guillaume.piolle.fr/}}
\newcommand{\maintainermail}{guillaume.piolle@centralesupelec.fr}
\newcommand{\currentVersion}{0.0}
  
\title{\Huge Gestion de versions\\Introduction à Git}
\author{\Large \maintainer\\\maintainerurl\\\texttt{\maintainermail}\\}
\date{}
% TODO date, version and revision on front page

\maketitle

\thanks{

  \par\vspace*{\fill}

  \paragraph*{Source}

  Ce document a été initialement publié sur la plate-forme GitHub, à
  l'adresse suivante~:
  \centerline{\url{https://github.com/Eusebius/IntroductionGit}}

  \paragraph*{Version}

  \currentVersion\
  (commit \gitAbbrevHash\ du \gitCommitterDate)
  
  \paragraph*{Licence}

  Ce document est publié sous la licence \textit{Creative Commons
    Attribution - Partage dans les Mêmes Conditions 3.0 France} (CC
  BY-SA 3.0 FR). Le texte complet peut en être trouvé ici~:
  \centerline{\url{https://creativecommons.org/licenses/by-sa/3.0/fr/legalcode}}

  C'est une licence libre au sens de la \textit{Free Software
    Foundation}, qui autorise quiconque à réutiliser ce document pour
  quelque usage que ce soit, y compris les usages commerciaux et la
  création d'\oe uvres dérivées, à condition que celles-ci soient
  elles-mêmes publiées sous une licence compatible.

  \paragraph*{Comment contribuer à ce document~?}

  N'importe qui peut proposer une contribution directe à ce document
  sous la forme d'un \textit{pull request} sur le dépôt GitHub de la
  publication initiale. Le ou les gestionnaires du dépôt décident
  seuls de l'intégration des propositions dans le dépôt principal.

  Les commentaires, suggestions d'évolution, signalements de coquilles
  ou d'erreurs\ldots peuvent être directement soumis sur GitHub sous
  forme d'\textit{issues}.

  En vertu de la licence utilisée, n'importe qui peut également créer
  une version librement dérivée (\textit{fork}) de ce document, à
  condition de la publier sous une licence compatible.

  \paragraph*{Liste des contributeurs}
  Les personnes physiques suivantes ont contribué à la conception de
  ce document, librement et en l'absence d'instructions d'aucune
  sorte~; elles sont les seuls détenteurs des droits d'auteur
  associés à ce document~:
  \begin{itemize}
  \item Guillaume Piolle \texttt{<guillaume.piolle@centralesupelec.fr>}.
  \end{itemize}

  \begin{versions}
    v0.0 & GP & 22/11/13
  \end{versions}
}

\frontmatter

\chapter*{Avant-propos}
\thispagestyle{empty}

TODO

% TODO avant-propos, objectifs et philosophie du document, utilisation
% de termes anglais

\mainmatter
\pagestyle{headings}
\setcounter{page}{1}
\tableofcontents
  
\chapter{Introduction non facultative} %TODO
\section{La gestion de versions} %TODO
% TODO lien avec le développement, aspects collaboratifs, fichiers
% texte et fichiers binaires
\section{Systèmes de gestion de version} %TODO
\subsection{Systèmes centralisés} %TODO
% TODO exemples : CVS, Subversion, Visual SourceSafe, Perforce Helix,
% VSTS (Visual Studio Team Services)
\subsection{Systèmes décentralisés} %TODO
% TODO exemples : Git, GNU Arch, Mercurial, BitKeeper, Bazaar, Darcs,
% Veracity, VSTS (Visual Studio Team Services)
\section{Le modèle de Git} %TODO
% TODO philosophie, différents éléments du modèle
\subsection{Principes et philosophie} %TODO
\subsection{Git sur les différents systèmes d'exploitation} %TODO
\subsection{Les différents éléments du modèle} %TODO
\subsection{Structure d'un dépôt} %TODO
% TODO notion de référence
\subsection{Modèle de configuration} %TODO
% TODO global / system / local, principales variables

\chapter{Le minimum à maîtriser pour survivre avec Git} % TODO
% TODO obtenir de l'aide
\section{Enregistrer et lister des modifications} %TODO
% TODO staging, commit, log, tags
\section{Gestion des fichiers non versionnés} % TODO
% TODO .gitignore
\section{Gestion des \textit{tags}}
% TODO navigation dans l'historique, reset, revert
% TODO branches, navigation entre branches
% TODO fusion et gestion des conflits
\section{Travailler avec un dépôt distant}
% TODO clone, push, pull

\chapter{Aspects plus avancés} % TODO
\section{Politiques de développement et de collaboration} %TODO
% TODO gestion des branches et des dépôts
\section{Le \textit{stash}} %TODO
% cherry-pick, réécriture d'historique, amend
\section{Stratégies de fusion} % TODO
\section{Remaniement de l'historique}
% TODO cherry-pick, réécriture d'historique, amend
\section{Sous-modules} % TODO
% TODO présenter brièvement les autres méthodes pour intégrer plusieurs projets

\chapter{Pratiques usuelles}
\section{Granularité des \textit{commits}} %TODO
\section{Versionnement sémantique} 
% TODO \cite{SemVer}
\section{Organisation de dépôts multiples}
% TODO notion de upstream et compagnie, différentes stratégies
% TODO collaboration par pull request
\section{Organisations des branches} %TODO
% TODO branches par version, par fonctionnalités, par sous-projets,
% gestion des tags...

\chapter{Lien avec d'autres logiciels et plates-formes} % TODO
% TODO bien préciser que les listes ne sont pas exhaustives
\section{Forges en ligne} % TODO
% TODO sourceforge, github, gitlab, bitbucket, gforge...
\section{Environnements de développement} % TODO
% TODO Eclipse, Netbeans
\section{Intégration continue et assurance qualité} % TODO
% TODO Code Climate, Jenkins, Travis...

\chapter{Les petites recettes pratiques qui sauvent la vie} % TODO
\section{Renoncer aux changements en cours} %TODO
% TODO les subtilités de reset
\section{Trouver l'origine d'un bug} % TODO
% TODO git blame, git bisect. Remonter dans la section sur l'historique ?
\section{Maintenir une différence constante entre deux branches} % TODO

\chapter{Glossaire} %TODO
\chapter{Index} %TODO

\nocite{*}

\bibliography{introgit}
\end{document}


%%% Local Variables:
%%% mode: latex
%%% TeX-master: t
%%% ispell-local-dictionary: "francais"
%%% End: 