 \chapter{Introduction non facultative}\label{chapIntro} %TODO
\section{La gestion de versions} %TODO
\index{Gestion de versions}
\index{commit@\textit{commit}}
% TODO lien avec le développement
% Suivi d'historique, retour en arrière. Notion de révision
% Collaboration à plusieurs, dépôts locaux et distants
% Développement de fonctionnalités en parallèle, fusion, résolution de conflits
% fichiers texte et fichiers binaires

\section{Systèmes de gestion de version} %TODO
\index{systeme de gestion de versions@système de gestion de versions}

\subsection{Systèmes centralisés} %TODO
\index{systeme de gestion de versions@système de gestion de versions!centralises@centralisé}
% TODO CVS
\index{CVS}
% TODO Subversion
\index{Subversion}
% TODO Visual SourceSafe
\index{Visual SourceSafe}
% TODO Perforce Helix
\index{Perforce Helix}
% TODO VSTS (Visual Studio Team Services)
\index{Visual Studio Team Services}

\subsection{Systèmes décentralisés} %TODO
\index{systeme de gestion de versions@système de gestion de versions!decentralises@décentralisé}
% TODO Git
\index{Git}
% TODO GNU arch
\index{GNU arch}
% TODO Mercurial
\index{Mercurial}
% TODO BitKeeper
\index{BitKeeper}
% TODO Bazaar
\index{Bazaar}
% TODO Darcs
\index{Darcs}
% TODO Veracity
\index{Veracity}
% TODO VSTS (Visual Studio Team Services)
\index{Visual Studio Team Services}

\section{Le modèle de Git} %TODO
% TODO philosophie, différents éléments du modèle

\subsection{Principes et philosophie} %TODO

\subsection{Git sur les différents systèmes d'exploitation}\label{GitOS} %TODO
\index{Windows}
\index{Linux}
\index{macOS}

\subsection{Les différents éléments du modèle} %TODO

\index{SHA1}
% TODO Expliquer en footnote que Git utilise SHA1 comme un code de
% détection d'erreur et non comme un algorithme de hachage
% cryptographique. En théorie, il est possible de falsifier le contenu
% d'un commit Git utilisant SHA1, par exemple pour y insérer du code
% malveillant, en conservant le même haché. Subversion utilise SHA1
% également, et les PDF en collision le faisaient planter, nécessitant
% un correctif. Sauf que Git utilise également SHA1 pour signer les
% commits, et là évidemment ça pose un problème d'authenticité et plus
% seulement d'intégrité.
% http://www.zdnet.fr/actualites/collision-sha-1-linus-torvalds-se-veut-rassurant-sur-git-39849070.htm
% https://twitter.com/matthew_d_green/status/835864260240683009/photo/1

\subsection{Structure d'un dépôt} %TODO
\index{depot@dépôt}
\index{depot@dépôt!local}
% TODO notion de référence

\subsection{Modèle de configuration} %TODO
\index{configuration}
\index{git@\texttt{git}!config@\texttt{config}}
\index{global@\texttt{global}}
\index{system@\texttt{system}}
\index{local@\texttt{local}}
% TODO global / system / local, principales variables