\chapter*{Avant-propos}\label{chapAvantPropos}
\thispagestyle{empty}

\section*{À qui s'adresse ce document~?}

Cet ouvrage a pour objectif de constituer une initiation à la gestion
de versions d'une manière générale, et à l'utilisation de Git en
particulier. Il est conçu comme un support pédagogique à l'usage
d'étudiants en informatique, ou de toute personne curieuse du sujet
ayant déjà une pratique raisonnable de l'outil informatique.

Le cas d'usage principal de Git et des autres systèmes de gestion de
versions est le développement logiciel en équipe. Néanmoins, il n'est
pas nécessaire d'être développeur pour tirer un bénéfice de
l'utilisation de Git. Les systèmes de gestion de versions peuvent par
exemple être utilisés pour gérer la rédaction de documents, comme
celui que vous avez sous les yeux\footnote{Et dont vous pouvez trouver
  le dépôt ici~: \url{https://github.com/Eusebius/IntroductionGit}.},
à condition que ces derniers soient manipulés en priorité sous un
format textuel (c'est-à-dire plutôt des fichiers \LaTeX
qu'OpenDocument, par exemple). Il devient particulièrement utile
lorsque plusieurs (voire de très nombreuses) personnes interviennent
dans le processus de conception. Cependant, les fonctionnalités des
systèmes de gestion de versions les rendent également très utiles à
des auteurs isolés souhaitant conserver et manipuler l'historique de
leur processus de création (en se ménageant par exemple la possibilité
de revenir à n'importe quel état antérieur ou de gérer concurremment
des variantes d'un même projet).

Git est à l'origine conçu pour le système d'exploitation Linux. Les
utilisateurs Linux ou macOS non réfractaires à la ligne de commande
pourront donc utiliser directement les commandes présentées ici. Que
les utilisateurs de Windows se rassurent, des outils et techniques ont
été développés très tôt pour permettre d'utiliser Git sur leurs
systèmes, et il y a maintenant de nombreuses manières de profiter de
Git sous Windows, au travers de logiciels dédiés ou bien directement
en ligne de commande. La section \ref{GitOS} donne quelques pistes
pour l'utilisation de Git sur ces systèmes\footnote{À ceux qui
  souhaiteraient utiliser Git sur des systèmes d'exploitation mobiles
  comme Android, iOS, Windows Phone ou Symbian, il convient de
  rappeler courtoisement que ceci est un ouvrage sérieux qui ne sied
  nullement à de telles pantalonnades.}.

Il est important de noter que cet ouvrage n'est pas une documentation
complète et exhaustive du logiciel Git et qu'il ne traite pas (loin
s'en faut) tous les sujets qui s'y rapportent. Ce document demeure une
introduction, qui pourra être complétée d'une part par la lecture des
ouvrages listés dans la bibliographie (en particulier le livre de
référence \textit{Pro Git} \cite{ProGit} accessible en ligne), d'autre
part et surtout par la pratique (et la confrontation à des cas
d'utilisation réels).

\section*{Comment lire ce document~?}

Un étudiant pressé de s'initier à une utilisation basique de Git, pour
les besoins d'un cours ou d'un projet limité dans le temps (ou
quiconque désirant une introduction rapide au sujet) pourra se
focaliser sur les chapitres \ref{chapIntro} et \ref{chapMinimum}, qui
ont pour objectif de lui mettre le pied à l'étrier et de le rendre
opérationnel au prix d'un effort de lecture et de pratique limité. Si
vous devez travailler sur Git avec un logiciel particulier, il est
possible que le chapitre \ref{chapLiens} comporte une section dédiée à
vos outils, vous proposant a minima des moyens de vous documenter sur
leur intégration avec Git.

Les étudiants libérés de l'urgence de l'évaluation et tous les
lecteurs suffisamment curieux et disponibles pourront alors compléter
leur découverte en parcourant le chapitre \ref{chapAvance}, qui
présente des fonctionnalités plus fouillées de Git, le chapitre
\ref{chapPratiques}, qui introduit certains usages et pratiques
couramment adoptés et le chapitre \ref{chapLiens} (en fonction de vos
besoins) sur l'intégration avec d'autres logiciels, plates-formes et
systèmes d'exploitation.

Le chapitre \ref{chapRecettes} a davantage vocation à devenir, au fur
et à mesure de l'évolution de cet ouvrage, un petit catalogue de
recettes, de patrons de conception, de \og trucs\fg bien pratiques
pour des situations problématiques ou des scénarios d'usage couramment
rencontrés.

\section*{Conventions utilisées}

Tout au long de cet ouvrage, nous adopterons les conventions
typographiques suivantes~:
\begin{itemize}
\item Les termes en \textit{italique} sont ceux qui proviennent d'une
  langue étrangère (l'anglais, sans surprise) mais qui restent
  utilisés tels quels dans le texte, soit que leur contrepartie en
  français soit moins usitée par les développeurs francophones, soit
  qu'il nous paraisse essentiel que le terme anglais soit connu, soit
  qu'ils correspondent à des commandes ou à des notions bien définies
  dans Git~;
\item Les termes en \textbf{gras} désignent des notions et concepts
  nouvellement introduits~;
\item Les termes en \texttt{chasse fixe} désignent des commandes, des
  options de configuration ou des noms de fichiers.
\end{itemize}